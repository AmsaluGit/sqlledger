\documentclass[a4paper,oneside]{article}
\usepackage[frame]{xy}
\usepackage{tabularx}
\usepackage[latin1]{inputenc}
\usepackage{marvosym}  % Euro \EUR
\usepackage{fancyhdr}
\setlength{\topmargin}{0cm}
\setlength{\topskip}{0cm}
\setlength{\headheight}{0cm}
\setlength{\headsep}{0.5cm}
\setlength{\textheight}{24.2cm}
\setlength{\textwidth}{19cm}
\setlength{\oddsidemargin}{-1.4cm}
\setlength{\evensidemargin}{-1.4cm}
\setlength{\footskip}{1cm}

\setlength{\parindent}{0pt}
\renewcommand{\baselinestretch}{1}
\begin{document}

\newlength{\descrwidth}\setlength{\descrwidth}{13.0cm}

\newsavebox{\hdr}

\fontfamily{cmss}\fontshape{n}\selectfont

\sbox{\hdr}{

\begin{minipage}[t]{0.6\linewidth}
\vspace{2.2cm}

\begin{tabular}[t]{p{1.7cm}p{2.4cm}p{1.7cm}}\\
\centering{N�mero} & \centering{Fecha} & C. Cliente\\
\centering{<%invnumber%>} & \centering{<%invdate%>} & \centering{<%customer_id%>}
\end{tabular}
\end{minipage}

\begin{minipage}[t]{0.4\linewidth}
\textbf{F A C T U R A} 
\vspace{1cm}

<%name%>

<%address1%>

<%address2%>

<%city%> <%state%> <%zipcode%>

<%country%>
\end{minipage}

}

\pagestyle{fancy}
\renewcommand{\headrulewidth}{0cm}
\renewcommand{\footrulewidth}{0cm}
\cfoot{\thepage}
%\markboth{\usebox{\hdr}}{\usebox{\hdr}}
%\thispagestyle{empty}     %use this with letterhead paper

<%pagebreak 65 27 37%>
\end{tabular*}
\newpage
\usebox{\hdr}
%\markboth{\usebox{\hdr}}{\usebox{\hdr}}
\vspace{0.5cm}

\begin{tabular*}{\textwidth}{rp{\descrwidth}rr}
  \textbf{Cant.} & \textbf{Descripci�n} & \textbf{Precio} & \textbf{Importe} \\ \hline
<%end pagebreak%>

\fontfamily{cmss}\fontsize{10pt}{12pt}\selectfont

\usebox{\hdr}
\vspace{0.5cm}

\begin{tabular*}{\textwidth}{rp{\descrwidth}rr}
  \textbf{Cant.} & \textbf{Descripci�n} & \textbf{Precio} & \textbf{Importe} \\ \hline
<%foreach number%>
  <%qty%> & <%description%> & <%sellprice%> & <%linetotal%> \\
<%end number%>
\end{tabular*}

\parbox{\textwidth}{
\vspace{12pt}
<%if notes%>
  <%notes%>
<%end if%>
}

\vfill

\begin{flushright}
\begin{tabularx}{10cm}{Xr@{}}
  \textbf{Base imponible} & \textbf{<%subtotal%>} \\
<%foreach tax%>
  IVA (<%taxrate%>\%) sobre <%taxbase%> & <%tax%> \\
<%end tax%>
  \hline
  \textbf{Total} & \textbf{<%invtotal%>} \\
\end{tabularx}
\end{flushright}

%\renewcommand{\thefootnote}{\fnsymbol{footnote}}

\end{document}

